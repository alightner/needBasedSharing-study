% Options for packages loaded elsewhere
\PassOptionsToPackage{unicode}{hyperref}
\PassOptionsToPackage{hyphens}{url}
%
\documentclass[
]{article}
\usepackage{amsmath,amssymb}
\usepackage{lmodern}
\usepackage{ifxetex,ifluatex}
\ifnum 0\ifxetex 1\fi\ifluatex 1\fi=0 % if pdftex
  \usepackage[T1]{fontenc}
  \usepackage[utf8]{inputenc}
  \usepackage{textcomp} % provide euro and other symbols
\else % if luatex or xetex
  \usepackage{unicode-math}
  \defaultfontfeatures{Scale=MatchLowercase}
  \defaultfontfeatures[\rmfamily]{Ligatures=TeX,Scale=1}
\fi
% Use upquote if available, for straight quotes in verbatim environments
\IfFileExists{upquote.sty}{\usepackage{upquote}}{}
\IfFileExists{microtype.sty}{% use microtype if available
  \usepackage[]{microtype}
  \UseMicrotypeSet[protrusion]{basicmath} % disable protrusion for tt fonts
}{}
\makeatletter
\@ifundefined{KOMAClassName}{% if non-KOMA class
  \IfFileExists{parskip.sty}{%
    \usepackage{parskip}
  }{% else
    \setlength{\parindent}{0pt}
    \setlength{\parskip}{6pt plus 2pt minus 1pt}}
}{% if KOMA class
  \KOMAoptions{parskip=half}}
\makeatother
\usepackage{xcolor}
\IfFileExists{xurl.sty}{\usepackage{xurl}}{} % add URL line breaks if available
\IfFileExists{bookmark.sty}{\usepackage{bookmark}}{\usepackage{hyperref}}
\hypersetup{
  pdftitle={In need-based sharing, need can be less important than sharing},
  pdfauthor={Aaron D. Lightner, Anne C. Pisor, Edward H. Hagen},
  hidelinks,
  pdfcreator={LaTeX via pandoc}}
\urlstyle{same} % disable monospaced font for URLs
\usepackage[margin=1in]{geometry}
\usepackage{longtable,booktabs,array}
\usepackage{calc} % for calculating minipage widths
% Correct order of tables after \paragraph or \subparagraph
\usepackage{etoolbox}
\makeatletter
\patchcmd\longtable{\par}{\if@noskipsec\mbox{}\fi\par}{}{}
\makeatother
% Allow footnotes in longtable head/foot
\IfFileExists{footnotehyper.sty}{\usepackage{footnotehyper}}{\usepackage{footnote}}
\makesavenoteenv{longtable}
\usepackage{graphicx}
\makeatletter
\def\maxwidth{\ifdim\Gin@nat@width>\linewidth\linewidth\else\Gin@nat@width\fi}
\def\maxheight{\ifdim\Gin@nat@height>\textheight\textheight\else\Gin@nat@height\fi}
\makeatother
% Scale images if necessary, so that they will not overflow the page
% margins by default, and it is still possible to overwrite the defaults
% using explicit options in \includegraphics[width, height, ...]{}
\setkeys{Gin}{width=\maxwidth,height=\maxheight,keepaspectratio}
% Set default figure placement to htbp
\makeatletter
\def\fps@figure{htbp}
\makeatother
\setlength{\emergencystretch}{3em} % prevent overfull lines
\providecommand{\tightlist}{%
  \setlength{\itemsep}{0pt}\setlength{\parskip}{0pt}}
\setcounter{secnumdepth}{5}
\usepackage{amssymb,amsmath}

\ifluatex
  \usepackage{selnolig}  % disable illegal ligatures
\fi
\newlength{\cslhangindent}
\setlength{\cslhangindent}{1.5em}
\newlength{\csllabelwidth}
\setlength{\csllabelwidth}{3em}
\newenvironment{CSLReferences}[2] % #1 hanging-ident, #2 entry spacing
 {% don't indent paragraphs
  \setlength{\parindent}{0pt}
  % turn on hanging indent if param 1 is 1
  \ifodd #1 \everypar{\setlength{\hangindent}{\cslhangindent}}\ignorespaces\fi
  % set entry spacing
  \ifnum #2 > 0
  \setlength{\parskip}{#2\baselineskip}
  \fi
 }%
 {}
\usepackage{calc}
\newcommand{\CSLBlock}[1]{#1\hfill\break}
\newcommand{\CSLLeftMargin}[1]{\parbox[t]{\csllabelwidth}{#1}}
\newcommand{\CSLRightInline}[1]{\parbox[t]{\linewidth - \csllabelwidth}{#1}\break}
\newcommand{\CSLIndent}[1]{\hspace{\cslhangindent}#1}

\title{In need-based sharing, need can be less important than sharing}
\author{Aaron D. Lightner, Anne C. Pisor, Edward H. Hagen}
\date{}

\begin{document}
\maketitle
\begin{abstract}
Need-based sharing pools risk by following two cooperative rules: help others when asked, and only request help when in need. In a two-part study, we first expanded an agent-based model of need-based sharing partnerships, adding defection and varying partnership sizes. We show that refusing to help always has a long-term cost, which increases with larger partnerships. In contrast, ``greedy'' requests that are not based on survival risk carry little-to-no cost. We then conducted an experimental vignette study of osotua, a need-based sharing tradition, with Tanzanian Maasai pastoralists. We found that participants generally complied with osotua requests, but shared larger amounts for requests that were based on survival risk. We conclude by proposing an expanded framework for need-based sharing, where the cost asymmetry among types of defection generally favors a decision heuristic where individuals prefer sharing with those in need, but err on the side of generosity when need is ambiguous.

{[}Abstract word count: 150, Main text word count: 4861{]}
\end{abstract}

Anthropologists have documented a variety of cultural institutions that manage risk through need-based sharing.\textsuperscript{1--6} A paradigmatic example of this is \emph{osotua}, a type of partnership where Maasai pastoralists share valuable resources with kin and non-kin, often without an apparent expectation of reciprocity.\textsuperscript{2,7--9} In Maa language, osotua literally translates to ``umbilical cord,'' metaphorically capturing how resources are freely shared from one individual to another based on need.\textsuperscript{10} Obligations to help an osotua partner are ``heavy'',\textsuperscript{2} and osotua can sometimes connote real and/or fictive kinship.\textsuperscript{7,9}

Formal models demonstrate that need-based sharing is mutually beneficial specifically because it pools risk among interdependent partners.\textsuperscript{11--15} Stated another way, agents benefit by keeping needy partners alive who are then able to return the favor in the future. Many of these models use dyadic partnerships for simplicity,\textsuperscript{16,17} although ethnographic data on need-based sharing institutions often indicate that partnerships may have multiple members and may be better conceptualized as network-level or even community-level institutions.\textsuperscript{8,9,18} Need-based sharing is a simple strategy, consisting of two rules for each individual to follow: (1) only ask for help when in need, and (2) when asked for help, share what you can if you are able.\textsuperscript{16,17} Hence, using the language of Claessens et al.,\textsuperscript{19} defection can occur either when recipients ask for help when they are not genuinely in need (\emph{greedy} defection), or when donors reject a request for help when they are able to share (\emph{stingy} defection). A recipient's ``genuine need'' typically refers to a relatively low probability of continued survival compared to the potential donor, usually due to resource scarcity.\textsuperscript{4,16,17,19--26} Because a reduced probability of survival implies a reduction in future fitness benefits, this formal definition makes need-based sharing relevant for understanding the evolution of cooperation.\textsuperscript{16,17,20,21}

And yet, this framing of ``genuine need'' as a reduced probability of survival is restrictive and might not always generalize to need-based sharing institutions in real-world settings. Consider, for example, a ``greedy'' defector who signals that they need help when their survival is not actually at risk. If they are able, a donor might minimize the prevalence of this cheating behavior by critically evaluating requests and deciding whether or not they should comply, basing their judgments on whether or not the signaler's request reflects a threat to survival (and hence, a threat to fitness). However, even when this is possible, the signaler can make a case to the donor for why they should be considered ``genuinely needy'' in a more colloquial sense. The donor's criteria for evaluating what does (or does not) constitute a convincing case can vary strongly across individuals, situations, and cultural contexts.\textsuperscript{27} Hence, in real-world scenarios, the criteria that a donor uses to evaluate a potential defector's need-based request might not reliably conform to the narrow survival criteria implied by formal need-based sharing models. Requests may be deemed acceptable in a given situation or culture, but, if they are mapped onto a need-based sharing model, deemed frivolous or greedy from a researchers' perspective.

To illustrate this more concretely, consider a polygynous herder who makes a need-based request to potential donor. In a cooperative scenario, the herder might state that he needs cattle after a drought, improving his (otherwise low) chances of survival and fitness. This request clearly conforms to formal definitions of need (e.g., in evolutionary models). Lacking subsistence-related resources due to unexpected misfortune is a risk to survival, so no greedy defection has occurred. We will refer to this justification for making a need-based request as an example of \emph{survival need}. In a more ``greedy'' scenario, the herder might instead say that he needs resources to pay bridewealth and acquire an additional spouse. Here, conferring resources would benefit the recipient's survival and fitness, despite the recipient lacking a relative survival disadvantage. In a formal model of need-based sharing, this behavior would be indistinguishable from greedy defection. We will refer to this justification for a need-based request as an example of \emph{non-survival need}. If a need-based sharing institution mandates that acceptable requests must not be greedy (i.e., they must be based on survival need), then a request made within that institution, if based on non-survival need, should be rejected.

The conundrum this creates is that different ways of \emph{conceptualizing} need might allow different levels of greedy behavior to occur, legitimizing requests based on non-survival need (possibly in addition to requests based on survival need). This could potentially undermine the long-term success of need-based sharing institutions. On the other hand, a key feature of need-based sharing institutions is that they are \emph{interdependent}, meaning that individuals should generally avoid cheating behaviors because undermining the success of one's partner also undermines one's own success over time.\textsuperscript{11,14,15} If a cultural institution bears the signs of interdependence, risk pooling, and need-based sharing, but if the individuals in that culture do not strictly enforce rule violations that will threaten its persistence (e.g., if a certain type of cheating behavior is culturally permissible), then this indicates that either the institution has failed, or that we should revise our assumptions about the consequences of apparently greedy behavior in need-based sharing.

This presents empirical questions about the boundary conditions for need-based sharing institutions: If these institutions pool risk \emph{because they are based on need},\textsuperscript{16} and if ``need'' refers to relative survival prospects, can greedy requests -- a type of defection, in formal need-based sharing models -- sometimes be socially permitted, simply because what counts as genuine need varies among different cultural contexts? If so, perhaps we should refine our understanding of need-based sharing or provide clarifications about how need-based sharing manages risk in real-world settings.

The present study seeks to answer these questions in the context of osotua among Maasai pastoralists using both modeling and empirical data. Osotua is often treated as the quintessential example of a need-based sharing institution,\textsuperscript{2,8,16,17,28} so by studying osotua we are returning to the roots of the need-based transfer literature to assess the relevance of survival need in sharing decisions.

In the modeling part of this study, we asked how the long-term benefits of need-based sharing might change as greedy and/or stingy defection become increasingly common, and how increasingly sizeable partnerships might influence these changes. We therefore assessed the boundary conditions of need-based sharing by extending an existing agent-based model (ABM) of osotua by Aktipis et al.~that examines survival outcomes among individuals who cooperate in need-based sharing partnerships.\textsuperscript{16,17} Their ABM found that need-based sharing can improve survival by pooling risk, and we asked how strictly individuals must adhere to the need-based sharing rules for these risk-pooling benefits to persist.

Specifically, in the original ABM, agents represented herders paired into dyads (\(N=2\) agents), each with the same initial number of livestock. In each time step (representing 1 year) each agent's herd was increased by the same fixed growth factor. Each herd was also subject to an environmental shock with low but non-zero probability that reduced it by a random but possibly substantial amount. When an agent's herd size fell below a fixed threshold, they made a need-based request to their partner for an amount equal to their deficit below the threshold. Partners shared the lesser of either the requested amount, or their surplus above the threshold. Agents whose herd sizes fell below the threshold for two consecutive time steps died. Hence, in this mutually cooperative scenario, each agent followed the rules of need-based sharing: (1) request resources only if you are in need, and (2) if you receive a request, share what you can without becoming needy. In this model, a greater number of agents survived at each time step compared to a model with no transfers.

Aktipis et al.\textsuperscript{16} also investigated models with ``probabilistic asking,'' in which a fixed request for cattle is made at a fixed rate, regardless of need (and there are no need-based requests), and ``probabilistic giving'', in which a fixed amount is provided in response to any request, regardless of the giver's herd size. In these models using these rules, survival was similar to or worse than a model with no exchange.

We extended and modified the original need-based sharing ABM in two ways. First, we included two possible defection strategies. The \emph{greedy strategy} violates the first need-based sharing rule by feigning survival need. This strategy differs from the Aktipis et al.\textsuperscript{16} ``probabilistic asking'' rule in that {[}greedy{]} requests are made by agents without need but {[}survival{]} requests are still made by agents in need. The \emph{stingy strategy} violates the second rule by refusing to share livestock upon request. This strategy differs substantially from the Aktipis et al.~``probabilistic giving'' rule in that transfers are still made according to need-based sharing rules, but with a transfer probability \(p<1\). Specifically, at each time step, each agent below the threshold made a need-based request, as in the original model, but agents above the threshold could commit each type of rule violation based on a probability parameter that was set prior to each simulation run (\(p_G\) for greedy defection and \(p_S\) for stingy defection). These parameters dictated how all agents in the partnership would behave. That is, in a given simulation run, each agent might be a pure strategy defector (\(p_S=1\) and/or \(p_G=1\)), or each agent could be a mixed strategy defector (\(p_S<1\) and/or \(p_G<1\)). The amounts of greedy requests were randomly drawn from the empirical distribution of need-based requests.

Second, and unlike Aktipis et al.,\textsuperscript{16} we also investigated larger completely connected partnerships, \(2 \le N \le 16\). We assessed each strategy's long-term survival after \(t=50\) years -- roughly a single generation.\footnote{Although this is an approximation that we make to conveniently estimate a medium-long timeframe, this generational equivalence is based on an assumption that household control of cattle is taken on around junior elderhood (about 30 years old) until death, where a typical lifespan is around 80 years.} The higher the fraction of agents surviving, the greater the strategy's \emph{survival} outcome.\textsuperscript{12,16,17} See the Methods for details, and \href{https://github.com/alightner/needBasedSharing-study}{github.com/alightner/needBasedSharing-study} for code.

In the empirical part of this study, we used field data to examine how important survival need-based requests are among osotua partners in a real-world setting, and to assess the extent to which osotua is a dyadic partnership vs.~a community-level institution. We tested a preregistered hypothesis that, when given an osotua request, Maasai participants would respond more generously to requests based on survival need than they would to requests based on non-survival need (analogous to the greedy requests in the ABM). In an experimental vignette, we asked 218 Tanzanian Kisongo Maasai pastoralists how many cows they would share with an osotua partner requesting assistance. Participants were randomly assigned to a condition where the request was either based on \emph{survival need} (resource scarcity after a drought; N = 116) or a greedy request based on \emph{non-survival need} (accumulating cattle to pay for bridewealth for a second wife; N = 102).\footnote{We specifically used \emph{bridewealth for a second wife} in the non-survival need condition because although key informants stated that a first wife was important for having a family with children, more than one wife, while acceptable, is far less common.} (See \href{https://osf.io/ndfxc}{osf.io/ndfxc} for preregistration materials.)

\section*{Results}

\subsection*{The agent-based model: Assessing boundary conditions in need-based sharing}

In the ABM, we first compared the survival among mutually cooperative pairs (\(p_S=0\) and \(p_G=0\)), purely greedy pairs (\(p_G=1\) and \(p_S=0\)), and purely stingy pairs (\(p_G=0\) and \(p_S=1\)). The mutually cooperative pairs clearly outperformed the stingy pairs (i.e., where no sharing occurs), replicating a key result of Aktipis et al.\textsuperscript{16} (Figure \ref{fig:abmplot0}). The substantial disadvantage incurred by stingy defectors is a straightforward consequence of the interdependence of need-based sharing partnerships: If defectors fail to share with their cooperative partners, then they will gain a short-term benefit while risking the loss of their partners. In the long run, this loss is a steep cost to defectors because they will require aid in the future.\textsuperscript{11,13--15} In contrast to purely stingy defectors, purely greedy defectors suffered little-to-no survival costs in the long term, and had similar survival outcomes when compared to the standard condition. The reason is that the \(p_G=1\) condition, in which need-based transfers still occur, is about as effective at reducing variance as the \(p_S=0\) and \(p_G=0\) condition. A condition with only greedy requests and no need-based requests, on the other hand, had higher variance and lower survival than any other condition (Figure \ref{fig:abmplot0}).

\begin{figure}
\centering
\includegraphics{needBasedSharing-paper_files/figure-latex/abmplot0-1.pdf}
\caption{\label{fig:abmplot0}Mean survival (top) and standard variation in herd sizes (bottom) over time. Left: group size = 2. Right: group size = 16. Colors represent models. Standard: The Aktipis et al.~(2011) ABM. Greedy: non-survival requests made in addition to need-based requests. Greedy without need-based requests: only greedy {[}non-survival based{]} requests are made (no need-based requests). Stingy: Need-based requests are refused with a probability of 0.5. The dashed line at t=50 shows where we calculated the survival advantage (or disadvantage) of partnerships.}
\end{figure}

We then re-ran this simulation across different partnership sizes (\(N\)) and \(p_G\) and \(p_S\) values. When we increased \(N\) for mutually cooperative partnerships (\(p_S=0\) and \(p_S=0\)), we saw higher survival outcomes. Further, and similar to the \(N=2\) results in Figure \ref{fig:abmplot0}, we saw that increasingly frequent greedy requests (\(p_G \rightarrow 1\) while \(p_S=0\)) had little-to-no impact on survival outcomes relative to the mutually cooperative outcomes. Hence, holding \(p_S\) to zero, increasing \(N\) substantially increased survival among all \(p_G \in [0,1]\). See Figure \ref{fig:abmplot1}. In contrast, increasing \(p_S\) from 0 to 1 reliably and substantially decreased survival across \(N\) values. Further, although survival when \(p_S=0\) increased as \(N\) increased, above moderate levels of stinginess, as partnership size increased from \(N=2\) to \(N=16\), survival outcomes were quicker to asymptote to \(\sim 25\%\) of agents surviving to \(t=50\). In other words, stingy defection becomes increasingly costly at a larger \(N\), because the difference in survival between \(p_S=0\) and \(p_S=1\) is positively associated with \(N\). See Figure \ref{fig:abmplot1}.

\begin{figure}
\centering
\includegraphics{needBasedSharing-paper_files/figure-latex/abmplot1-1.pdf}
\caption{\label{fig:abmplot1}Average survival outcomes after 50 time steps (color) across 100,000 simulation runs for each configuration of partnership sizes (N, shown here by facets), probability of stingy defection (y-axis), and probability of greedy defection (x-axis).}
\end{figure}

Taken together, these ABM results identify a critical caveat about need-based sharing: An interdependent need-based sharing institution like osotua is generally tolerant to greedy defection, but stingy defection always undermines its long-term sustainability. Further, because need-based sharing improves survival in increasingly large partnerships (even when greedy defectors are present), the relative importance of generosity, compared to detecting and preventing greedy requests, should be higher among larger need-based sharing partnerships. Put differently, stinginess becomes more costly as need-based sharing institutions scale to larger, more interconnected groups.

With a paradigmatic need-based sharing institution such as osotua, we can only resolve the typical size of a partnership and the extent to which Maasai tolerate non-need-based requests with empirical data. To what extent will osotua partners respond to need-based vs.~non-need-based requests with generosity? Are osotua partnerships restricted to dyads, or are they community-based, where we would expect greedy defection to be more likely to be tolerated?

\subsection*{The experimental vignette: Testing the effect of need in Maasai osotua}\label{experiment}

We read vignettes to Tanzanian Maasai pastoralists in which an osotua partner requested assistance based on survival need (need-based request) or non-survival need. We found that survival need increased the amount shared (\(\beta\) = 0.42, \(p\) = 0.044, \(SE\) = 0.21), controlling for household food insecurity,\textsuperscript{29} a proxy for reduced sharing ability (\(\beta\) = 0.014, \(p\) = 0.96, \(SE\) = 0.3). Simplifying our model by not controlling for food insecurity -- which deviates from our preregistered analysis plan -- did not alter the positive causal effect of survival need on amount shared (\(\beta\) = 0.42, \(p\) = 0.043, \(SE\) = 0.21).

Nevertheless, most participants were willing to share small numbers of cattle even when survival need was absent. Despite amounts shared being slightly higher for need-based requests, both conditions elicited similar levels of generosity (drought: mean = 2.8, median = 3, sd = 1.3; bridewealth: mean = 2.4, median = 2, sd = 1.7; Mann-Whitney U = 7507, \(p\) = \ensuremath{1.8\times 10^{-4}}), and out of the 218 participants in this experiment, only 1 participant refused to share anything. See Figure \ref{fig:amtshareplot}.

\begin{figure}
\centering
\includegraphics{needBasedSharing-paper_files/figure-latex/amtshareplot-1.pdf}
\caption{\label{fig:amtshareplot}Empirical cumulative distribution of the reported amounts of cattle that participants were willing to share, by experimental condition. Amounts were out of an initial resource holding of twenty head of cattle.}
\end{figure}

In our exploratory analysis, we assessed the scale of osotua partnerships using qualitative and observational data, interview data from key informants, and social network data from each participant. Qualitative and observational data suggested that osotua is a large-scale, community-level institution: when participants were asked with whom they had osotua, a common response was ``everyone.'' This was not surprising after our initial key informant interviews, where people referred to osotua as a perspective, a personality characteristic akin to generosity and kindness, and/or ``comfort.'' According to some informants, examples of ``comfort'' included helpful acts during times of distress, such as offering a Maasai stranger food and shelter when s/he is far from home, or helping a warrior calm his nerves before his circumcision ceremony. As a first pass, these ethnographic data suggest that osotua is indeed a broad institution (i.e., many connected to many) for Kisongo Maasai.

The participants' views of osotua as a community-level phenomenon made collecting social network data about osotua partnerships challenging. We addressed this challenge by asking participants, ``with whom do you have \emph{the most} osotua?'' Consistent with our initial observations, many individuals had many relatively close osotua partners, even after restricting their responses to their closest and most salient partners (Figure \ref{fig:degosotuaplot}).\footnote{Many of the names given during the social network data collection phase had included mentions that were not included in our sample. These represent the boundaries of our social network, and this is a primary reason why we focus here on the number of ties per participant rather than a complete sociocentric network.}

\begin{figure}
\centering
\includegraphics{needBasedSharing-paper_files/figure-latex/degosotuaplot-1.pdf}
\caption{\label{fig:degosotuaplot}Counts of the number of mentioned names that each participant gave when asked with whom he or she had the most osotua.}
\end{figure}

Our main results showed a positive effect of need on sharing, and as we and others have showed, need-based sharing in osotua \emph{can} reduce individual livelihood risks. To evaluate whether or not this might actually be the case, we analyzed the impact of having more vs.~fewer close osotua partners on household food insecurity, finding a curvilinear association between number of osotua partners and food insecurity scores. Consistent with risk pooling accounts of need-based sharing, food insecurity was negatively associated with the number of close osotua ties among individuals with fewer ties (number of osotua partners: \(\beta=\) -0.056, \(p=\) \ensuremath{3.2\times 10^{-4}}, \(SE\) = 0.015; number of osotua partners squared: \(\beta=\) 0.0021, \(p=\) 0.0088, \(SE\) = \ensuremath{8.1\times 10^{-4}}; see Figure \ref{fig:osotuainsecure}).

\begin{figure}
\centering
\includegraphics{needBasedSharing-paper_files/figure-latex/osotuainsecure-1.pdf}
\caption{\label{fig:osotuainsecure}Regression model of household food insecurity scores as a function of number of osotua partners, based on a social network of close osotua partnerships. Larger points indicate a greater number of participants with a given osotua degree and level of food insecurity.}
\end{figure}

\section*{Discussion}\label{discussion}

In this study we investigated Maasai osotua, a paradigmatic example of a need-based sharing institution that reduces individuals' risk, improving long-term survival prospects.\textsuperscript{17,30,31} A challenge for need-based sharing institutions, which many other cooperative institutions also face, is that societies must somehow motivate individuals to cooperate while reducing temptations to defect. Need-based sharing has two potential sources of defection --- refusing to help someone when they ask for help (the stingy strategy) and signaling need when genuine livelihood risks are not present (the greedy strategy) --- and existing models have not considered each separately. Yet, a vulnerability of need-based sharing is the potential ambiguity between cooperative vs.~greedy requests: Individuals might colloquially justify their claims about ``genuine need'' in ways that do not conform to its standard uses in formal and evolutionary models, which tend to narrowly focus on risks to continued survival.

In an agent-based model (ABM), we extended the ABM of Aktipis et al.,\textsuperscript{16,17} a dyadic model without defection, to both expand the size of need-based partnerships (e.g.,)\textsuperscript{12,32,33} and to distinguish greedy defection from stingy defection. Distinguishing greedy from stingy defection, something not done in previous models, suggests that cooperation in need-based sharing is generally important for long-term risk pooling outcomes that are mutually beneficial to participants' survival. These outcomes are resilient against greedy defection, but not against stingy defection (Figure \ref{fig:abmplot0}). Further, increasing the size of need-based sharing partnerships also increases the benefits of risk pooling, and as they increase in size, these partnerships remain resilient to greedy defection. On the other hand, stingy defection is increasingly costly as partnership sizes increase (Figure \ref{fig:abmplot1}). We therefore identify a subtlety in the boundary conditions of need-based sharing institutions: Assuming interdependence, greediness is easily tolerated but stinginess is not -- and it becomes increasingly costly as partnerships increase in size.

We can consider the behavioral implications of this asymmetry between greedy vs.~stingy behavior using error management theory\textsuperscript{34,35} and the multiplicative growth of assets---both cattle herds and networks of cooperators.\textsuperscript{36--39} Consider the decision task for need-based sharing partners in an institution like osotua. Each agent takes on a scenario-specific role at each time step: a potential \emph{recipient} who can send a signal for aid, and a potential \emph{donor} who receives that signal. At each time step, the recipient sends a potentially ambiguous signal, which was either generated by a state of need or a state of ``greed.'' Under such ambiguity, the expected cost of stinginess toward needy requests (\(p_S >0\) and \(p_G=0\) in the ABM) always outweighs the expected cost of sharing with greedy requests (\(p_G>0\) and \(p_S=0\) in the ABM; see Figure \ref{fig:abmplot1}), because the cost of rejecting a genuine request is high. To understand this cost, first consider the cost to the herd of rejecting a genuine request. Because of the multiplicative nature of herd growth, rare shocks are amplified, meaning each individual's resource level is constantly trending toward zero.\textsuperscript{36,40} Interdependence reduces the risk of hitting zero.\textsuperscript{11,13,14,33,36,41,42} Second, individuals have to stay alive to help one another in future instances of need, but because of the interdependence fostered by need-based institutions, shocks again have multiplicative effects. Assuming greedy individuals are willing to help in future instances of need (i.e., when \(p_S \approx 0\)), then when greedy defectors supplement their need-based requests with non-need-based ones, they improve their own chances of surviving long enough to help each other in future instances of need. Taken to its extreme of \(p_G=1\), and assuming that individuals are willing to share when asked to, the greedy strategy essentially becomes a group-level pooling strategy, which is also effective for reducing variance around annual growth rates (and potential losses).\textsuperscript{36,43}

Results from our empirical field study among Tanzanian Kisongo Maasai pastoralists were consistent with these insights. Supporting a need-based sharing account of osotua, we found that participants were willing to share more cattle when an osotua request was based on survival need (feeding one's family after a drought) rather than non-survival need (obtaining bridewealth for a second wife). We also found that, for most people, that individuals who were more involved in osotua partnerships had lower household food insecurity scores -- with the exception of a few individuals with very large numbers of osotua partners, reflected in the curvilinear relationship (Figures \ref{fig:amtshareplot} and \ref{fig:osotuainsecure}). This exception might reflect a high demand among individuals with many osotua partnerships, which we will discuss further below. Importantly, though, participants frequently complied with non-survival need-based, or ``greedy,'' requests. Our interview data also suggested that osotua is an important and broad, community-level institution, and generosity in osotua was frequently described as something that should be applied to other Maasai people across many facets of life. Informants gave examples of how and when osotua can occur beyond formal cattle sharing partnerships -- like about how it can even include behaviors such as hospitality toward a traveling stranger. During social network data collection, even when we restricted participants' responses to the most salient and important osotua ties, participants reported widespread community involvement in osotua: many participants have many close ties (Figure \ref{fig:degosotuaplot}).

Taken together, our models and experimental vignette findings suggest a broad, testable framework that expands on the work of Aktipis, Cronk, and others. While our results support the claim that osotua is a need-based sharing institution that beneficially pools risk,\textsuperscript{11,16,17,33} the asymmetry between costs of greed vs.~stinginess shows that sharing is generally favored, even when greedy requests might be prevalent. In future research we would expect that in institutions with need-based sharing and interdependence, people err on the side of sharing, rather than rejecting, a potentially frivolous request. We would also expect this to tendency to increase as environments become harsher and more volatile, and as partnerships increase in size and levels of perceived interdependence.\textsuperscript{41,42} Specifically, we would predict that need-based sharing among many densely connected individuals is more tolerant of non-need-based requests (as we observed in our field study with Tanzanian Maasai pastoralists), compared to rigid, dyadic partnerships (e.g., in a sparse and/or controlled social network).\textsuperscript{44--46} Conversely, while we would expect that generosity would be more central to need-based sharing institutions than restraint from making non-need-based requests, we would expect restraint to become more important when greedy requests become excessively frequent, and when environments and subsistence strategies become more stable.

Our results reframe ongoing conversations about the nature of human sociality, specifically with respect to how ``need'' works on the ground, how osotua works, and how need-based sharing is or is not distinct from delayed reciprocity. First, evolutionary models tend to characterize ``need'' in terms of continued survival prospects, and as we show, some culturally acceptable ways of justifying need are formally equivalent to the greedy strategy -- even in osotua, the touchstone example of need-based sharing. Future research could apply a similar approach to other evolutionary theories by investigating the extent to which theoretical constructs in our formal models actually map onto folk constructs ``on the ground'' in a given cultural context (e.g.,).\textsuperscript{47}

Second, our findings suggest that for Kisongo Maasai, osotua is less akin to a formal partnership and more akin to a broad, community-level need-based sharing ``ethos'' in which non-need-based requests are acceptable. It is nevertheless conceivable that need-based sharing in other cultural contexts \emph{is less} forgiving of non-need-based requests. Cronk,\textsuperscript{2} for example, found that among the Mukugodo Maasai, non-need-based requests among osotua partners were ``unthinkable'' (p.~353). Our findings do not contradict Cronk's, but complement them: Mukogodo Maasai of Kenya are geographically quite distant from the Kisongo Maasai of northern Tanzania, and their histories are very different.\textsuperscript{48,49} Based on the theoretical framework that we have laid out here, osotua for Mukogodo Maasai might expand from a history of demand sharing as former hunter-gatherers (rather than account-keeping; see also),\textsuperscript{50} possibly with a social structure consisting of far less redundancy among sharing network ties than those observed among Kisongo Maasai.

Third, a compelling avenue for future research would be to investigate how need-based sharing and delayed reciprocity are related but differentially expressed as a consequence of different social and environmental contexts. Our findings resemble key characteristics of delayed reciprocity, where future returns are somehow expected, but strict book-keeping is not.\textsuperscript{6} Indeed, during fieldwork, multiple informants told ADL a common Kisongo Maasai saying about osotua: \emph{engiteng osotua elo eshukunyie} (translated from Maa: ``whatever cow I give in osotua will someday come back'' -- if not from one osotua partner, from another). Higher vs.~lower levels of interdependence might account for how people deem non-need-based requests as unacceptable vs.~acceptable, and community-level need-based sharing practices that we observed in this study also seemed to bear key similarities to the demand sharing practices, for example, that have been documented among some forager groups.\textsuperscript{51--53} In such cases, requests are frequent and not necessarily based on need, but an expectation of generosity compels people to share small amounts. In future work, a more general theory of sharing as risk reduction could unify these apparently separate modes of cooperation.\textsuperscript{50,54,55}

Our study has a few limitations. The ABM, for simplicity, assumes that increasingly large partnerships are completely connected communities. Network structure can play an influential role in need-based sharing,\textsuperscript{12,32} but we currently lack the empirical data to create a clear and detailed picture of how osotua networks are structured at the society level. Additionally, our ABM does not determine if the strategies we investigate can evolve over the course of multiple generations.

Our field evidence that individuals with more close osotua partners had lower household food insecurity was not a preregistered hypothesis, but a post hoc exploratory finding, and therefore has a higher risk of being spurious. Additionally, whereas we interpret the result as more support from osotua translating to lower-risk livelihoods, it might instead suggest that fewer resources translates to lower partner value, leading to fewer osotua partners. However, this interpretation would conflict with the observation that osotua partnerships often originate in childhood, and can even be passed down to one's children.\textsuperscript{2,9}

Further, we did not systematically investigate how people subjectively approved or disapproved of the requests in the experimental vignette. We thus inferred participants' approval based on their sharing behavior, although our key informants suggested that the bridewealth requests were neither taboo nor inappropriate. We also did not test the extent to which people prioritized sharing unconditionally, without expectation of future help, or how important people thought it was to be generous rather than stingy in the osotua interaction we described in the vignette. Our key informant interviews and observational evidence did suggest that generosity was extremely important, but because costly stinginess was so central to our theoretical results, future research should directly test our inferences that rules against greedy defection should be less stringent than rules against stingy defection.

To conclude, evolutionary theories of cooperation suggest that need-based sharing is a critical part of human cultures everywhere.\textsuperscript{4,56} These theories suggest that need-based sharing practices follow two simple rules: help others when asked, and only ask for help when genuine need arises. Formally, genuine need is typically defined in terms of long-term survival risks, but cross-culturally and colloquially, its criteria might vary. In an experimental vignette, we investigated how and why Tanzanian Kisongo Maasai pastoralists respond to requests in osotua, a paradigmatic example of need-based sharing, that were either based on survival need (needing subsistence after a drought) or non-survival need (needing bridewealth to acquire a second wife). We found that requests based on survival need were weakly associated with more cattle sharing than requests based on non-survival need. Sharing was widespread across both conditions, however, and participants were almost always willing to share small amounts. In an ABM, we showed why this might be the case: interdependent need-based sharing partnerships can easily tolerate greediness, but they do not tolerate stinginess -- especially in large, interconnected partnerships like osotua among the Kisongo Maasai.

\section*{Methods}\label{themethods}

\subsection*{The agent-based model}\label{abmmethods}

In our replication of Aktipis et al.,\textsuperscript{16,17} agents represent herders with a livestock value, \(x\), that is necessary for its continuing viability. Viability, as we discuss further below, relies on agents consistently staying above a minimum herd size of \(x_{min} = 64\). While we extended the model to include multiple partnership sizes (\(2 \le N \le 16\)) of completely connected networks of agents (as we discuss further below), for the initial replication of Aktipis et al., we paired each agent into a dyadic partnership (\(N=2\)) at the initialization of each simulation. Each agent at each time step, \(t\), followed a basic protocol consisting of 4 phases: growth, shock, transfer, and viability checks. The parameters we selected came from Aktipis et al.\textsuperscript{16,17} unless otherwise stated. As we outline next, the protocol for growth, shock, and viability check phases is identical in the replication and our extension of Aktipis et al., but the transfer phase is not.

At initialization, each agent (in this simulation, all agents are \emph{cooperative agents}) starts with a herd size of \(x_{0}=70\) livestock. At each time step, the growth phase is when an agent's viability \(x_{i,t+1}\) is increased based on its viability in the previous time step and a random, normally distributed growth rate \(g_{i,t} \sim \mathcal{N}(0.034, 0.0253)\), where \(x_{i,t+1} = x_{i,t} + x_{i,t}g_{i,t}\). (Note that this distribution implies that ``negative growth'' is possible, but rare.) The shock phase represents the possibility of a sudden and unexpected livestock loss that disrupts the general upward growth trend. Losses occur at each time step for each agent with probability of 0.1, applying a random loss magnitude \(l_{i,t} \sim \mathcal{N}(0.3, 0.1)\) to the agent's livestock level, where \(x_{i,t+1} =x_{i,t} - x_{i,t}l_{i,t}\).

The transfer phase is when need-based sharing occurs. The transfer phase in Aktipis et al.~is mutually cooperative, because each agent follows the rules of need-based sharing: (1) make a request of your dyadic partner for resources only if you are in need (\(x_{i,t} < x_{min}\)), and (2) if you receive a request, share what you can out of your own surplus (i.e., transfer the amount that a partner requested, up to \(x_{i,t}-x_{min}\)). If a cooperator is in need (\(x_{i,t} < x_{min}\)), then it will request the amount needed, \(x_{min} - x_{i,t}\), from its partner. If a cooperator receives a request from its partner, and if it is able to share a surplus without becoming needy (\(x_{i,t} > x_{min}\)), then it will share the amount requested up to its own minimum viability threshold (\(x_{i,t} - x_{min}\)). If an agent receives a request but is below the viability threshold, then it is unable to share and will therefore transfer nothing. Notice here that if both partners are below the minimum viability threshold at the same time, they can send requests to each other but neither will transfer (because both are unable).

The viability check then determines whether or not an agent has gone extinct from the simulation. An agent's number of livestock \(x_{i,t}\) is checked against a lower viability threshold \(x_{min}\), and if two consecutive time steps have passed below this threshold (i.e., \(x_{i,t} < x_{min}\) and \(x_{i,t-1} < x_{min}\)), then the agent \(i\) is removed from the simulation. Once an agent is removed from the simulation, any agents partnered with it are no longer able to make requests from, or interact with, that agent.

Our extension of Aktipis et al.~adds two possible defection strategies, each of which is based on the agent's tendency to violate one of the two rules in the transfer phase. Each possible defection strategy was set to a probability of occurring at each time step, \(p\), at initialization prior to each simulation run. This probability applied to all agents in each simulation run, i.e., there were no simulations where some agents were more or less likely to defect than other agents.

Agents using the \emph{greedy strategy} violate the first rule of need-based sharing by feigning need at a given time step. A greedy agent, if it is in need (\(x_{i,t} < x_{min}\)), will make a need-based request as described above. However, if a greedy agent is not in need at a given time step, it will feign need with a probability \(p_G\) and request an amount that was drawn from a sample of amounts requested (\(x_{min} - x_{i,t}\)) that were requested in previous runs of the standard condition. This ensured that greedy requests were not disproportionately small or large compared to survival need-based requests.

Agents using the \emph{stingy strategy} violate the second rule of need-based sharing by refusing to share livestock upon request. Specifically, if a need-based request occurs at a given time step, a stingy agent will either refuse to share with probability \(p_S\), or comply with the request and share as a cooperator with probability \(1-p_S\). (Note that like cooperative agents, stingy agents can still make need-based requests and refuse to share upon request when they are in need themselves, i.e., if \(x_{i,t} < x_{min}\).) The probability of defection at each time step was set at initialization by the parameter \(p_S\); both \(p_S\) and, for the greedy strategy, \(p_G\) were identical for all defectors in a given simulation run (including in simulations where there were more than two agents in a partnership, as we discuss next).

For simulations involving \(N>2\) agents, each request was made of one random agent in the same group, regardless of that agent's need or other requests directed at that agent. If a request was refused, no further request was made. An agent might receive multiple requests, and could transfer cattle for each request.

Agents inevitably reach extinction after some number of discrete time steps (instead of replicating), so our outcome of interest was a relative extinction rate, or the \emph{survival} per time step. Survival was measured by the fraction of the population of agents still viable at a given time step. This fraction was out of the \(100\,000\) simulation runs that we ran for each parameter set. A \emph{parameter set} refers to a specific combination of partnership size (\(N\)) and the probability of defection per time step (\(p_G\) and \(p_S\)). Specifically, the parameter sets for each type of defection and partnership size were combinations across partnership sizes \(N \in \{ 2, 3, \dots, 16 \}\), greedy defection probability \(p_G \in \{0, 0.1, 0.2, \dots, 1\}\), and stingy defection probability \(p_S \in \{0, 0.1, 0.2, \dots, 1\}\). After running \(100\,000\) simulations for each parameter set, we measured the long-term survival at \(t=50\) time steps. (Note that this is not strictly consistent with Aktipis et al.,\textsuperscript{17} who averaged survival outcomes over \(10\,000\) simulations. We increased this number to ensure reproducibility, though the difference from \(10\,000\) runs was negligible.)

All data and code can be found at \href{https://github.com/alightner/needBasedSharing-study}{github.com/alightner/needBasedSharing-study}.

\subsection*{The experimental vignette}\label{experimentmethods}

We conducted an experimental vignette with Kisongo Maasai pastoralists in Monduli Juu highlands of northern Tanzania (N = 218, 41.3\% female). We asked how each participant would respond to a request for assistance from a person with whom they had osotua. Prior to the vignette, each participant was given a structured survey that included measures of household food insecurity and social network data. The vignette and survey were administered by ADL with a Maasai translator or by one of two Maasai research assistants. All participants were paid 10,000 TZS (about \$4.35) for their participation, and all protocols and survey materials were approved by Washington State University IRB and Tanzanian Commission for Science and Technology (COSTECH) prior to data collection.

The experimental vignette described a hypothetical Maasai person with whom the participant had osotua, who has requested cattle to help him during a difficult time. The participant in this scenario was told that s/he had twenty cows at the time.\footnote{A fixed number of cattle was included in the vignette to keep the allocation numbers comparable between participants. Twenty cattle was specifically chosen because this was deemed a realistic number based on observation data collected by ADL during fieldwork. Herds in this region are often disaggregated across households, families, and friends, and frequent trading, selling, and buying means that herd sizes are relatively fluid from year to year. Basically, the hypothetical scenario outlined to participants was designed to be as realistic as possible to as many participants as possible.} Participants were randomly assigned to a condition in which this request was either based on \emph{survival need}, because it involved resource scarcity after a drought, or \emph{non-survival need}, because it involved someone accumulating cattle to pay bridewealth for a second wife (N = 116 in the drought condition, N = 102 in the bridewealth condition). We refer to this manipulated variable as the presence/absence of \emph{need}, where need refers to a condition that would cause relatively low chances of continued survival.

Consistent with the rules of a Dictator game, in which participants are given a resource allocation and must decide how much of it to share, participants were asked how many of their cows they would share with the person in the vignette requesting help. This number of cows shared, between 0 and 20, was the outcome measure of interest, which we refer to as \emph{shared}. The structured survey included household food insecurity scores,\textsuperscript{29} which we refer to as \emph{insecure}. Because the rules of need-based sharing include a caveat where individuals must share if they are able to, our confirmatory models controlled for food insecurity scores, a proxy for reduced sharing ability. Hence, the linear regression model used in our confirmatory analysis was

\[
\text{share} = \alpha + \beta_1 \times \text{need} + \beta_2 \times \text{insecure}
\]

where we predicted that \(\beta_1 > 0\), \(p < 0.05\). We used social network data for exploratory analyses. Social network data consisted of egocentric networks, created after asking participants to list the people with whom they had the most osotua. As discussed in the main text, we collected social network data by asking participants with whom they had the most osotua, which allowed us to count the number of people mentioned per participant. Participants frequently volunteered more information about osotua during this part of the survey, and we collected these statements for additional context.

Preregistration materials can be found at \href{https://osf.io/ndfxc}{osf.io/ndfxc}, and the anonymized experimental and structured survey data can be found at \href{https://github.com/alightner/needBasedSharing-study}{github.com/alightner/needBasedSharing-study}.

\section*{Acknowledgements}

We especially thank the Maasai people who participated in this study. We also thank Musa Kamaika, Kotoke Ngilepoi, and an anonymous Maasai research assistant for invaluable aid during the field study. We acknowledge funding from a National Science Foundation Doctoral Dissertation Improvement Grant (award number 1918523), and ADL acknowledges funding from the Aarhus University Research Foundation.

\section*{References}

\hypertarget{refs}{}
\begin{CSLReferences}{0}{0}
\leavevmode\hypertarget{ref-lee1979kung}{}%
\CSLLeftMargin{1. }
\CSLRightInline{Lee, R. B. \emph{The !kung san: Men, women and work in a foraging society}. (Cambridge University Press, 1979).}

\leavevmode\hypertarget{ref-cronkInfluenceCulturalFraming2007}{}%
\CSLLeftMargin{2. }
\CSLRightInline{Cronk, L. The influence of cultural framing on play in the trust game: A {Maasai} example. \emph{Evolution and Human Behavior} \textbf{28}, 352--358 (2007).}

\leavevmode\hypertarget{ref-cronkSolidarityTypeWorldNeedBased2021}{}%
\CSLLeftMargin{3. }
\CSLRightInline{Cronk, L., Guevara Beltrán, D., Mercado, D. L. \& Aktipis, A. {`{A Solidarity}-{Type World}'}: {Need}-{Based Helping} among {Ranchers} in the {Southwestern United States}. \emph{Human Nature} (2021) doi:\href{https://doi.org/10.1007/s12110-021-09406-8}{10.1007/s12110-021-09406-8}.}

\leavevmode\hypertarget{ref-hruschkaFriendshipDevelopmentEcology2010}{}%
\CSLLeftMargin{4. }
\CSLRightInline{Hruschka, D. J. \emph{Friendship: {Development}, ecology, and evolution of a relationship}. vol. 5 ({Univ of California Press}, 2010).}

\leavevmode\hypertarget{ref-smithBenefitsCostlySignaling2003a}{}%
\CSLLeftMargin{5. }
\CSLRightInline{Smith, E. A., Bird, R. B. \& Bird, D. W. The benefits of costly signaling: {Meriam} turtle hunters. \emph{Behavioral Ecology} \textbf{14}, 116--126 (2003).}

\leavevmode\hypertarget{ref-wiessnerHuntingHealingHxaro2002}{}%
\CSLLeftMargin{6. }
\CSLRightInline{Wiessner, P. Hunting, healing, and hxaro exchange: {A} long-term perspective on !{Kung} ({Ju}/'hoansi) large-game hunting. \emph{Evolution and Human Behavior} \textbf{23}, 407--436 (2002).}

\leavevmode\hypertarget{ref-jacobsTraditionalPoliticalOrganization1965}{}%
\CSLLeftMargin{7. }
\CSLRightInline{Jacobs, A. The traditional political organization of the pastoral {Maasai}. (University of California - Los Angeles, 1965).}

\leavevmode\hypertarget{ref-perlovTradingInfluenceSocial1987}{}%
\CSLLeftMargin{8. }
\CSLRightInline{Perlov, D. C. Trading for influence: {The} social and cultural economics of livestock marketing among the highland {Samburu} of {Northern Kenya}. (University of California - Los Angeles, 1987).}

\leavevmode\hypertarget{ref-spencerSamburuStudyGerontocracy1965}{}%
\CSLLeftMargin{9. }
\CSLRightInline{Spencer, P. \emph{The {Samburu}: {A Study} of {Gerontocracy} in a {Nomadic Tribe}}. ({Routledge}, 1965).}

\leavevmode\hypertarget{ref-hollisNoteMasaiSystem1910}{}%
\CSLLeftMargin{10. }
\CSLRightInline{Hollis, A. C. A {Note} on the {Masai System} of {Relationship} and {Other Matters Connected Therewith}. \emph{The Journal of the Royal Anthropological Institute of Great Britain and Ireland} \textbf{40}, 473--482 (1910).}

\leavevmode\hypertarget{ref-aktipisUnderstandingCooperationFitness2018}{}%
\CSLLeftMargin{11. }
\CSLRightInline{Aktipis, A. \emph{et al.} Understanding cooperation through fitness interdependence. \emph{Nature Human Behaviour} \textbf{2}, 429--431 (2018).}

\leavevmode\hypertarget{ref-kayserSocialOptimaNeedbased2016}{}%
\CSLLeftMargin{12. }
\CSLRightInline{Kayser, K. \& Armbruster, D. Social {Optima} of {Need}-based {Transfers}. (2016).}

\leavevmode\hypertarget{ref-kelleyInterpersonalRelationsTheory1978}{}%
\CSLLeftMargin{13. }
\CSLRightInline{Kelley, H. H. \& Thibaut, J. W. \emph{Interpersonal relations: A theory of interdependence}. ({Wiley}, 1978).}

\leavevmode\hypertarget{ref-robertsCooperationInterdependence2005}{}%
\CSLLeftMargin{14. }
\CSLRightInline{Roberts, G. Cooperation through interdependence. \emph{Animal Behaviour} \textbf{70}, 901--908 (2005).}

\leavevmode\hypertarget{ref-tomaselloTwoKeySteps2012}{}%
\CSLLeftMargin{15. }
\CSLRightInline{Tomasello, M., Melis, A. P., Tennie, C., Wyman, E. \& Herrmann, E. Two {Key Steps} in the {Evolution} of {Human Cooperation}: {The Interdependence Hypothesis}. \emph{Current Anthropology} \textbf{53}, 673--692 (2012).}

\leavevmode\hypertarget{ref-aktipisRiskPoolingHerdSurvival2011}{}%
\CSLLeftMargin{16. }
\CSLRightInline{Aktipis, C. A., Cronk, L. \& Aguiar, R. de. Risk-{Pooling} and {Herd Survival}: {An Agent}-{Based Model} of a {Maasai Gift}-{Giving System}. \emph{Human Ecology} \textbf{39}, 131--140 (2011).}

\leavevmode\hypertarget{ref-aktipisCooperationUncertainWorld2016}{}%
\CSLLeftMargin{17. }
\CSLRightInline{Aktipis, A. \emph{et al.} Cooperation in an {Uncertain World}: {For} the {Maasai} of {East Africa}, {Need}-{Based Transfers Outperform Account}-{Keeping} in {Volatile Environments}. \emph{Human Ecology} \textbf{44}, 353--364 (2016).}

\leavevmode\hypertarget{ref-spearBeingMaasaiEthnicity1993}{}%
\CSLLeftMargin{18. }
\CSLRightInline{\emph{Being {Maasai}: Ethnicity \& identity in {East Africa}}. ({Currey {[}u.a.{]}}, 1993).}

\leavevmode\hypertarget{ref-claessensNeedbasedTransferSystems2021}{}%
\CSLLeftMargin{19. }
\CSLRightInline{Claessens, S., Ayers, J. D., Cronk, L. \& Aktipis, A. Need-based transfer systems are more vulnerable to cheating when resources are hidden. \emph{Evolution and Human Behavior} \textbf{42}, 104--112 (2021).}

\leavevmode\hypertarget{ref-aktipisathenaPrinciplesCooperationSystems2015}{}%
\CSLLeftMargin{20. }
\CSLRightInline{Aktipis Athena. Principles of cooperation across systems: From human sharing to multicellularity and cancer. \emph{Evolutionary Applications} \textbf{9}, 17--36 (2015).}

\leavevmode\hypertarget{ref-eshelPartnership2001}{}%
\CSLLeftMargin{21. }
\CSLRightInline{Eshel, I. \& Shaked, A. Partnership. \emph{Journal of Theoretical Biology} \textbf{208}, 457--474 (2001).}

\leavevmode\hypertarget{ref-garayCooperationDefencePredator2009}{}%
\CSLLeftMargin{22. }
\CSLRightInline{Garay, J. Cooperation in defence against a predator. \emph{Journal of Theoretical Biology} \textbf{257}, 45--51 (2009).}

\leavevmode\hypertarget{ref-johnstoneContinuousSirPhilip1992}{}%
\CSLLeftMargin{23. }
\CSLRightInline{Johnstone, R. A. \& Grafen, A. The continuous {Sir Philip Sidney} game: {A} simple model of biological signalling. \emph{Journal of Theoretical Biology} \textbf{156}, 215--234 (1992).}

\leavevmode\hypertarget{ref-kayserEconomicsNeedbasedTransfers2018}{}%
\CSLLeftMargin{24. }
\CSLRightInline{Kayser, K. The economics of need-based transfers. (Arizona State University, 2018).}

\leavevmode\hypertarget{ref-maynard-smithAnimalSignals2003}{}%
\CSLLeftMargin{25. }
\CSLRightInline{Maynard-Smith, J. \& Harper, D. \emph{Animal {Signals}}. ({Oxford University Press}, 2003).}

\leavevmode\hypertarget{ref-wilkinsonReciprocalFoodSharing1984}{}%
\CSLLeftMargin{26. }
\CSLRightInline{Wilkinson, G. S. Reciprocal food sharing in the vampire bat. \emph{Nature} \textbf{308}, 181--184 (1984).}

\leavevmode\hypertarget{ref-mercierEnigmaReason2017}{}%
\CSLLeftMargin{27. }
\CSLRightInline{Mercier, H. \& Sperber, D. \emph{The enigma of reason}. ({Harvard University Press}, 2017).}

\leavevmode\hypertarget{ref-cronkUnfamiliarSocialNorm2008}{}%
\CSLLeftMargin{28. }
\CSLRightInline{Cronk, L. \& Wasielewski, H. An unfamiliar social norm rapidly produces framing effects in an economic game. \emph{Journal of Evolutionary Psychology} \textbf{6}, 283--308 (2008).}

\leavevmode\hypertarget{ref-blumbergEffectivenessShortForm1999}{}%
\CSLLeftMargin{29. }
\CSLRightInline{Blumberg, S. J., Bialostosky, K., Hamilton, W. L. \& Briefel, R. R. The effectiveness of a short form of the {Household Food Security Scale}. \emph{American Journal of Public Health} \textbf{89}, 1231--1234 (1999).}

\leavevmode\hypertarget{ref-campenniCorrelatedDisastersNeedbased2017}{}%
\CSLLeftMargin{30. }
\CSLRightInline{Campenni, M., Cronk, L. \& Aktipis, A. Correlated disasters and need-based transfers: {The} limits of risk pooling systems in simulated ecologies. (2017) doi:\href{https://doi.org/10.1101/230607}{10.1101/230607}.}

\leavevmode\hypertarget{ref-cronk2021design}{}%
\CSLLeftMargin{31. }
\CSLRightInline{Cronk, L. \& Aktipis, A. Design principles for risk-pooling systems. \emph{Nature Human Behaviour} \textbf{5}, 825--833 (2021).}

\leavevmode\hypertarget{ref-haoNeedbasedTransfersNetwork2015}{}%
\CSLLeftMargin{32. }
\CSLRightInline{Hao, Y., Armbruster, D., Cronk, L. \& Aktipis, C. A. Need-based transfers on a network: A model of risk-pooling in ecologically volatile environments. \emph{Evolution and Human Behavior} \textbf{36}, 265--273 (2015).}

\leavevmode\hypertarget{ref-campenni2021need}{}%
\CSLLeftMargin{33. }
\CSLRightInline{Campenni, M., Cronk, L. \& Aktipis, A. Need-based transfers enhance resilience to shocks: An agent-based model of a maasai risk-pooling system. \emph{Human Ecology} 1--14 (2021).}

\leavevmode\hypertarget{ref-haseltonAdaptiveRationalityEvolutionary2009a}{}%
\CSLLeftMargin{34. }
\CSLRightInline{Haselton, M. \emph{et al.} Adaptive rationality: {An} evolutionary perspective on cognitive bias. \emph{Psychology Faculty Articles and Research} (2009).}

\leavevmode\hypertarget{ref-mckay2010subtleties}{}%
\CSLLeftMargin{35. }
\CSLRightInline{McKay, R. \& Efferson, C. The subtleties of error management. \emph{Evolution and human behavior} \textbf{31}, 309--319 (2010).}

\leavevmode\hypertarget{ref-priceFitnessmaximizersEmployPessimistic2020}{}%
\CSLLeftMargin{36. }
\CSLRightInline{Price, M. H. \& Jones, J. H. Fitness-maximizers employ pessimistic probability weighting for decisions under risk. \emph{Evolutionary Human Sciences} \textbf{2}, (2020).}

\leavevmode\hypertarget{ref-dyson-hudsonNomadicPastoralism1980}{}%
\CSLLeftMargin{37. }
\CSLRightInline{Dyson-Hudson, R. \& Dyson-Hudson, N. Nomadic pastoralism. \emph{Annual review of anthropology} \textbf{9}, 15--61 (1980).}

\leavevmode\hypertarget{ref-lesorogolAssetBuildingCommunity2009}{}%
\CSLLeftMargin{38. }
\CSLRightInline{Lesorogol, C. K. Asset building through community participation: Restocking pastoralists following drought in northern {Kenya}. \emph{Social Work in Public Health} \textbf{24}, 178--186 (2009).}

\leavevmode\hypertarget{ref-bolligRiskManagementHazardous2010}{}%
\CSLLeftMargin{39. }
\CSLRightInline{Bollig, M. \emph{Risk management in a hazardous environment: {A} comparative study of two pastoral societies}. vol. 2 ({Springer Science \& Business Media}, 2010).}

\leavevmode\hypertarget{ref-petersEvaluatingGamblesUsing2016}{}%
\CSLLeftMargin{40. }
\CSLRightInline{Peters, O. \& Gell-Mann, M. Evaluating gambles using dynamics. \emph{Chaos: An Interdisciplinary Journal of Nonlinear Science} \textbf{26}, 023103 (2016).}

\leavevmode\hypertarget{ref-ballietFunctionalInterdependenceTheory2017}{}%
\CSLLeftMargin{41. }
\CSLRightInline{Balliet, D., Tybur, J. M. \& Van Lange, P. A. M. Functional {Interdependence Theory}: {An Evolutionary Account} of {Social Situations}. \emph{Personality and Social Psychology Review} \textbf{21}, 361--388 (2017).}

\leavevmode\hypertarget{ref-gerpottHowPeopleThink2018}{}%
\CSLLeftMargin{42. }
\CSLRightInline{Gerpott, F. H., Balliet, D., Columbus, S., Molho, C. \& de Vries, R. E. How do people think about interdependence? {A} multidimensional model of subjective outcome interdependence. \emph{Journal of Personality and Social Psychology} \textbf{115}, 716--742 (2018).}

\leavevmode\hypertarget{ref-peters2022ergodicity}{}%
\CSLLeftMargin{43. }
\CSLRightInline{Peters, O. \& Adamou, A. The ergodicity solution of the cooperation puzzle. \emph{Philosophical Transactions of the Royal Society A} \textbf{380}, 20200425 (2022).}

\leavevmode\hypertarget{ref-aktipis2004know}{}%
\CSLLeftMargin{44. }
\CSLRightInline{Aktipis, C. A. Know when to walk away: Contingent movement and the evolution of cooperation. \emph{Journal of theoretical biology} \textbf{231}, 249--260 (2004).}

\leavevmode\hypertarget{ref-baumard2013mutualistic}{}%
\CSLLeftMargin{45. }
\CSLRightInline{Baumard, N., André, J.-B. \& Sperber, D. A mutualistic approach to morality: The evolution of fairness by partner choice. \emph{Behavioral and Brain Sciences} \textbf{36}, 59--78 (2013).}

\leavevmode\hypertarget{ref-pisor2018diversify}{}%
\CSLLeftMargin{46. }
\CSLRightInline{Pisor, A. C. \& Gurven, M. When to diversify, and with whom? Choosing partners among out-group strangers in lowland bolivia. \emph{Evolution and Human Behavior} \textbf{39}, 30--39 (2018).}

\leavevmode\hypertarget{ref-stibbard2022hunt}{}%
\CSLLeftMargin{47. }
\CSLRightInline{Stibbard-Hawkes, D. N., Smith, K. \& Apicella, C. L. Why hunt? Why gather? Why share? Hadza assessments of foraging and food-sharing motive. \emph{Evolution and Human Behavior} \textbf{43}, 257--272 (2022).}

\leavevmode\hypertarget{ref-cronkMukogodoMaasaiEthnicity2004}{}%
\CSLLeftMargin{48. }
\CSLRightInline{Cronk, L. \emph{From {Mukogodo} to {Maasai}: Ethnicity and cultural change in {Kenya}}. ({Westview Press}, 2004).}

\leavevmode\hypertarget{ref-spearBeingMaasaiEthnicity1993a}{}%
\CSLLeftMargin{49. }
\CSLRightInline{\emph{Being {Maasai}: {Ethnicity} and {Identity In East Africa}}. ({Ohio University Press}, 1993).}

\leavevmode\hypertarget{ref-iyer2016risk}{}%
\CSLLeftMargin{50. }
\CSLRightInline{Iyer, K. P. Risk management through social networks among male and female pastoralists in karamoja, uganda. (Rutgers University-Graduate School-New Brunswick, 2016).}

\leavevmode\hypertarget{ref-peterson1993demand}{}%
\CSLLeftMargin{51. }
\CSLRightInline{Peterson, N. Demand sharing: Reciprocity and the pressure for generosity among foragers. \emph{American anthropologist} \textbf{95}, 860--874 (1993).}

\leavevmode\hypertarget{ref-blurton1987tolerated}{}%
\CSLLeftMargin{52. }
\CSLRightInline{Blurton Jones, N. G. Tolerated theft, suggestions about the ecology and evolution of sharing, hoarding and scrounging. \emph{Social Science Information} \textbf{26}, 31--54 (1987).}

\leavevmode\hypertarget{ref-jones1984selfish}{}%
\CSLLeftMargin{53. }
\CSLRightInline{Jones, N. B. A selfish origin for human food sharing: Tolerated theft. \emph{Ethology and sociobiology} \textbf{5}, 1--3 (1984).}

\leavevmode\hypertarget{ref-winterhalder1996marginal}{}%
\CSLLeftMargin{54. }
\CSLRightInline{Winterhalder, B. A marginal model of tolerated theft. \emph{Ethology and Sociobiology} \textbf{17}, 37--53 (1996).}

\leavevmode\hypertarget{ref-bird1997delayed}{}%
\CSLLeftMargin{55. }
\CSLRightInline{Bird, R. L. B. \& Bird, D. W. Delayed reciprocity and tolerated theft: The behavioral ecology of food-sharing strategies. \emph{Current anthropology} \textbf{38}, 49--78 (1997).}

\leavevmode\hypertarget{ref-cronkManagingRiskCooperationinpress}{}%
\CSLLeftMargin{56. }
\CSLRightInline{Cronk, L. \emph{et al.} Managing risk through cooperation: {Need}-based transfers and risk pooling among the societies of the {Human Generosity Project}. in \emph{Global perspectives on long term community resource management} ({Springer}, 2019).}

\end{CSLReferences}

\end{document}
